% Options for packages loaded elsewhere
\PassOptionsToPackage{unicode}{hyperref}
\PassOptionsToPackage{hyphens}{url}
%
\documentclass[
]{article}
\usepackage{lmodern}
\usepackage{amssymb,amsmath}
\usepackage{ifxetex,ifluatex}
\ifnum 0\ifxetex 1\fi\ifluatex 1\fi=0 % if pdftex
  \usepackage[T1]{fontenc}
  \usepackage[utf8]{inputenc}
  \usepackage{textcomp} % provide euro and other symbols
\else % if luatex or xetex
  \usepackage{unicode-math}
  \defaultfontfeatures{Scale=MatchLowercase}
  \defaultfontfeatures[\rmfamily]{Ligatures=TeX,Scale=1}
\fi
% Use upquote if available, for straight quotes in verbatim environments
\IfFileExists{upquote.sty}{\usepackage{upquote}}{}
\IfFileExists{microtype.sty}{% use microtype if available
  \usepackage[]{microtype}
  \UseMicrotypeSet[protrusion]{basicmath} % disable protrusion for tt fonts
}{}
\makeatletter
\@ifundefined{KOMAClassName}{% if non-KOMA class
  \IfFileExists{parskip.sty}{%
    \usepackage{parskip}
  }{% else
    \setlength{\parindent}{0pt}
    \setlength{\parskip}{6pt plus 2pt minus 1pt}}
}{% if KOMA class
  \KOMAoptions{parskip=half}}
\makeatother
\usepackage{xcolor}
\IfFileExists{xurl.sty}{\usepackage{xurl}}{} % add URL line breaks if available
\IfFileExists{bookmark.sty}{\usepackage{bookmark}}{\usepackage{hyperref}}
\hypersetup{
  pdftitle={CFA},
  pdfauthor={Matthew Blanchard},
  hidelinks,
  pdfcreator={LaTeX via pandoc}}
\urlstyle{same} % disable monospaced font for URLs
\usepackage[margin=1in]{geometry}
\usepackage{color}
\usepackage{fancyvrb}
\newcommand{\VerbBar}{|}
\newcommand{\VERB}{\Verb[commandchars=\\\{\}]}
\DefineVerbatimEnvironment{Highlighting}{Verbatim}{commandchars=\\\{\}}
% Add ',fontsize=\small' for more characters per line
\usepackage{framed}
\definecolor{shadecolor}{RGB}{248,248,248}
\newenvironment{Shaded}{\begin{snugshade}}{\end{snugshade}}
\newcommand{\AlertTok}[1]{\textcolor[rgb]{0.94,0.16,0.16}{#1}}
\newcommand{\AnnotationTok}[1]{\textcolor[rgb]{0.56,0.35,0.01}{\textbf{\textit{#1}}}}
\newcommand{\AttributeTok}[1]{\textcolor[rgb]{0.77,0.63,0.00}{#1}}
\newcommand{\BaseNTok}[1]{\textcolor[rgb]{0.00,0.00,0.81}{#1}}
\newcommand{\BuiltInTok}[1]{#1}
\newcommand{\CharTok}[1]{\textcolor[rgb]{0.31,0.60,0.02}{#1}}
\newcommand{\CommentTok}[1]{\textcolor[rgb]{0.56,0.35,0.01}{\textit{#1}}}
\newcommand{\CommentVarTok}[1]{\textcolor[rgb]{0.56,0.35,0.01}{\textbf{\textit{#1}}}}
\newcommand{\ConstantTok}[1]{\textcolor[rgb]{0.00,0.00,0.00}{#1}}
\newcommand{\ControlFlowTok}[1]{\textcolor[rgb]{0.13,0.29,0.53}{\textbf{#1}}}
\newcommand{\DataTypeTok}[1]{\textcolor[rgb]{0.13,0.29,0.53}{#1}}
\newcommand{\DecValTok}[1]{\textcolor[rgb]{0.00,0.00,0.81}{#1}}
\newcommand{\DocumentationTok}[1]{\textcolor[rgb]{0.56,0.35,0.01}{\textbf{\textit{#1}}}}
\newcommand{\ErrorTok}[1]{\textcolor[rgb]{0.64,0.00,0.00}{\textbf{#1}}}
\newcommand{\ExtensionTok}[1]{#1}
\newcommand{\FloatTok}[1]{\textcolor[rgb]{0.00,0.00,0.81}{#1}}
\newcommand{\FunctionTok}[1]{\textcolor[rgb]{0.00,0.00,0.00}{#1}}
\newcommand{\ImportTok}[1]{#1}
\newcommand{\InformationTok}[1]{\textcolor[rgb]{0.56,0.35,0.01}{\textbf{\textit{#1}}}}
\newcommand{\KeywordTok}[1]{\textcolor[rgb]{0.13,0.29,0.53}{\textbf{#1}}}
\newcommand{\NormalTok}[1]{#1}
\newcommand{\OperatorTok}[1]{\textcolor[rgb]{0.81,0.36,0.00}{\textbf{#1}}}
\newcommand{\OtherTok}[1]{\textcolor[rgb]{0.56,0.35,0.01}{#1}}
\newcommand{\PreprocessorTok}[1]{\textcolor[rgb]{0.56,0.35,0.01}{\textit{#1}}}
\newcommand{\RegionMarkerTok}[1]{#1}
\newcommand{\SpecialCharTok}[1]{\textcolor[rgb]{0.00,0.00,0.00}{#1}}
\newcommand{\SpecialStringTok}[1]{\textcolor[rgb]{0.31,0.60,0.02}{#1}}
\newcommand{\StringTok}[1]{\textcolor[rgb]{0.31,0.60,0.02}{#1}}
\newcommand{\VariableTok}[1]{\textcolor[rgb]{0.00,0.00,0.00}{#1}}
\newcommand{\VerbatimStringTok}[1]{\textcolor[rgb]{0.31,0.60,0.02}{#1}}
\newcommand{\WarningTok}[1]{\textcolor[rgb]{0.56,0.35,0.01}{\textbf{\textit{#1}}}}
\usepackage{graphicx,grffile}
\makeatletter
\def\maxwidth{\ifdim\Gin@nat@width>\linewidth\linewidth\else\Gin@nat@width\fi}
\def\maxheight{\ifdim\Gin@nat@height>\textheight\textheight\else\Gin@nat@height\fi}
\makeatother
% Scale images if necessary, so that they will not overflow the page
% margins by default, and it is still possible to overwrite the defaults
% using explicit options in \includegraphics[width, height, ...]{}
\setkeys{Gin}{width=\maxwidth,height=\maxheight,keepaspectratio}
% Set default figure placement to htbp
\makeatletter
\def\fps@figure{htbp}
\makeatother
\setlength{\emergencystretch}{3em} % prevent overfull lines
\providecommand{\tightlist}{%
  \setlength{\itemsep}{0pt}\setlength{\parskip}{0pt}}
\setcounter{secnumdepth}{-\maxdimen} % remove section numbering
\usepackage{caption}
\captionsetup{labelsep = newline}
\captionsetup{justification = centering, singlelinecheck = false}
\usepackage{pdflscape}
\newcommand{\blandscape}{\begin{landscape}}
\newcommand{\elandscape}{\end{landscape}}
\usepackage{booktabs}
\usepackage{longtable}
\usepackage{array}
\usepackage{multirow}
\usepackage{wrapfig}
\usepackage{float}
\usepackage{colortbl}
\usepackage{pdflscape}
\usepackage{tabu}
\usepackage{threeparttable}
\usepackage{threeparttablex}
\usepackage[normalem]{ulem}
\usepackage{makecell}
\usepackage{xcolor}

\title{CFA}
\author{Matthew Blanchard}
\date{}

\begin{document}
\maketitle

\hypertarget{missing-value-analysis}{%
\section{Missing value analysis}\label{missing-value-analysis}}

\includegraphics{cfa_files/figure-latex/missing-1.pdf}

Data are missing for 5 groups. I will impute missing values for now.

\begin{verbatim}
## `summarise()` regrouping output by 'group' (override with `.groups` argument)
## `summarise()` regrouping output by 'group' (override with `.groups` argument)
\end{verbatim}

\begin{table}[H]

\caption{\label{tab:miss}Number of missing values for each group}
\centering
\fontsize{12}{14}\selectfont
\begin{tabular}[t]{lrr}
\toprule
group & member\_1 & member\_2\\
\midrule
19013010\_1 & 4 & 4\\
17102710\_2 & 2 & 2\\
18080915\_1 & 2 & 2\\
18110810\_2 & 2 & 2\\
18110713\_2 & 1 & 1\\
\bottomrule
\end{tabular}
\end{table}

\hypertarget{confirmatory-factor-analysis-for}{%
\section{Confirmatory Factor Analysis
for}\label{confirmatory-factor-analysis-for}}

\hypertarget{fit-one-factor-model}{%
\subsection{Fit one factor model}\label{fit-one-factor-model}}

\begin{Shaded}
\begin{Highlighting}[]
\NormalTok{cfa_one <-}\StringTok{ }\KeywordTok{paste0}\NormalTok{(}\StringTok{'}
\StringTok{                ci  =~ adr.ind.acc + crt.ind.acc + rapm.ind.acc +}
\StringTok{                  adr.ind.conf + crt.ind.conf + rapm.ind.conf'}\NormalTok{)}
\end{Highlighting}
\end{Shaded}

\begin{table}[H]

\caption{\label{tab:unnamed-chunk-2}Goodness of fit indices}
\centering
\fontsize{12}{14}\selectfont
\begin{tabular}[t]{lr}
\toprule
index & value\\
\midrule
chisq & 138.256\\
df & 9.000\\
pvalue & 0.000\\
gfi & 0.987\\
tli & 0.602\\
\addlinespace
cfi & 0.761\\
rmsea & 0.246\\
rmsea.ci.lower & 0.210\\
rmsea.ci.upper & 0.283\\
\bottomrule
\end{tabular}
\end{table}

Model fit is poor: CFI and TLI are low and RMSEA is too high. Let's try
a two-factor model with accuracy and confidence.

\hypertarget{fit-two-factor-model-accuracy-confidence}{%
\subsection{Fit two factor model (Accuracy +
Confidence)}\label{fit-two-factor-model-accuracy-confidence}}

\begin{Shaded}
\begin{Highlighting}[]
\NormalTok{cfa_two <-}\StringTok{ }\KeywordTok{paste0}\NormalTok{(}\StringTok{'}
\StringTok{                accuracy  =~ adr.ind.acc + crt.ind.acc + rapm.ind.acc }
\StringTok{                }
\StringTok{                confidence =~ adr.ind.conf + crt.ind.conf + rapm.ind.conf}
\StringTok{                     '}\NormalTok{)}
\end{Highlighting}
\end{Shaded}

\begin{table}[H]

\caption{\label{tab:unnamed-chunk-4}Goodness of fit indices}
\centering
\fontsize{12}{14}\selectfont
\begin{tabular}[t]{lr}
\toprule
index & value\\
\midrule
chisq & 138.256\\
df & 9.000\\
pvalue & 0.000\\
gfi & 0.987\\
tli & 0.602\\
\addlinespace
cfi & 0.761\\
rmsea & 0.246\\
rmsea.ci.lower & 0.210\\
rmsea.ci.upper & 0.283\\
\bottomrule
\end{tabular}
\end{table}

\begin{table}[H]

\caption{\label{tab:unnamed-chunk-5}Test model difference}
\centering
\fontsize{12}{14}\selectfont
\begin{tabular}[t]{lrrrrrrr}
\toprule
  & Df & AIC & BIC & Chisq & Chisq diff & Df diff & Pr(>Chisq)\\
\midrule
fit\_two & 8 & 12098.99 & 12164.97 & 114.5160 & NA & NA & NA\\
fit\_one & 9 & 12120.73 & 12183.23 & 138.2558 & 23.73981 & 1 & 1.1e-06\\
\bottomrule
\end{tabular}
\end{table}

The two-factor model is an improvement on the one-factor model.
Although, the model fit is still poor so we may be able to do better by
correlating accuracy and confidence within the same test.

\hypertarget{correlate-accuracy-and-confidence-within-each-test}{%
\subsection{Correlate accuracy and confidence within each
test}\label{correlate-accuracy-and-confidence-within-each-test}}

\hypertarget{fit-one-factor-model-correlated}{%
\subsubsection{Fit one factor model
(correlated)}\label{fit-one-factor-model-correlated}}

\begin{Shaded}
\begin{Highlighting}[]
\NormalTok{cfa_one_corr <-}\StringTok{ }\KeywordTok{paste0}\NormalTok{(}\StringTok{'}
\StringTok{                ci  =~ adr.ind.acc + crt.ind.acc + rapm.ind.acc +}
\StringTok{                        adr.ind.conf + crt.ind.conf + rapm.ind.conf}
\StringTok{                        }
\StringTok{                adr.ind.acc ~~ adr.ind.conf}
\StringTok{                crt.ind.acc ~~ crt.ind.conf}
\StringTok{                rapm.ind.acc ~~ rapm.ind.conf}
\StringTok{                '}\NormalTok{)}
\end{Highlighting}
\end{Shaded}

\begin{table}[H]

\caption{\label{tab:unnamed-chunk-7}Goodness of fit indices}
\centering
\fontsize{12}{14}\selectfont
\begin{tabular}[t]{lr}
\toprule
index & value\\
\midrule
chisq & 138.256\\
df & 9.000\\
pvalue & 0.000\\
gfi & 0.987\\
tli & 0.602\\
\addlinespace
cfi & 0.761\\
rmsea & 0.246\\
rmsea.ci.lower & 0.210\\
rmsea.ci.upper & 0.283\\
\bottomrule
\end{tabular}
\end{table}

Again, model fit is poor: CFI and TLI are low and RMSEA is too high.
Let's try the two-factor model.

\hypertarget{fit-two-factor-model-correlated}{%
\subsection{Fit two factor model
(correlated)}\label{fit-two-factor-model-correlated}}

\begin{Shaded}
\begin{Highlighting}[]
\NormalTok{cfa_two_corr <-}\StringTok{ }\KeywordTok{paste0}\NormalTok{(}\StringTok{'}
\StringTok{                accuracy  =~ adr.ind.acc + crt.ind.acc + rapm.ind.acc }
\StringTok{                confidence =~ adr.ind.conf + crt.ind.conf + rapm.ind.conf}

\StringTok{                adr.ind.acc ~~ adr.ind.conf}
\StringTok{                crt.ind.acc ~~ crt.ind.conf}
\StringTok{                rapm.ind.acc ~~ rapm.ind.conf}
\StringTok{                     '}\NormalTok{)}
\end{Highlighting}
\end{Shaded}

\begin{table}[H]

\caption{\label{tab:unnamed-chunk-9}Goodness of fit indices}
\centering
\fontsize{12}{14}\selectfont
\begin{tabular}[t]{lr}
\toprule
index & value\\
\midrule
chisq & 138.256\\
df & 9.000\\
pvalue & 0.000\\
gfi & 0.987\\
tli & 0.602\\
\addlinespace
cfi & 0.761\\
rmsea & 0.246\\
rmsea.ci.lower & 0.210\\
rmsea.ci.upper & 0.283\\
\bottomrule
\end{tabular}
\end{table}

\begin{table}[H]

\caption{\label{tab:unnamed-chunk-10}Test model difference}
\centering
\fontsize{12}{14}\selectfont
\begin{tabular}[t]{lrrrrrrr}
\toprule
  & Df & AIC & BIC & Chisq & Chisq diff & Df diff & Pr(>Chisq)\\
\midrule
fit\_two\_corr & 5 & 12002.46 & 12078.85 & 11.97908 & NA & NA & NA\\
fit\_one\_corr & 6 & 12087.61 & 12160.53 & 99.13595 & 87.15687 & 1 & 0\\
\bottomrule
\end{tabular}
\end{table}

The two-factor model appears to be a good fit for the data and it is
significantly better than the one-factor model. Yay!

\end{document}
